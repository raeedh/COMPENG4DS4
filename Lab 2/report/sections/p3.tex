\begin{filecontents}[overwrite]{./sections/p3_main.tex}
\begin{lstlisting}[language=c,caption=Problem 3 main, label=list:p3_main]
int main(void) {
    BaseType_t status;
    /* Init board hardware. */
    BOARD_InitBootPins();
    BOARD_InitBootClocks();
    BOARD_InitDebugConsole();

    usr_str = malloc(sizeof(char) * 100);

    SemaphoreHandle_t *semaphores = (SemaphoreHandle_t*) malloc(3 * sizeof(SemaphoreHandle_t));

    semaphores[0] = xSemaphoreCreateBinary(); // Producer1_sem
    semaphores[1] = xSemaphoreCreateBinary(); // Producer2_sem
    semaphores[2] = xSemaphoreCreateCounting(2, 2); // consumer_sem

    status = xTaskCreate(producer_sem, "producer", 200, (void*) semaphores, 2, NULL);
    if (status != pdPASS) {
        PRINTF("Task creation failed!.\r\n");
        while (1)
            ;
    }

    status = xTaskCreate(consumer1_sem, "consumer", 200, (void*) semaphores, 2, NULL);
    if (status != pdPASS) {
        PRINTF("Task creation failed!.\r\n");
        while (1)
            ;
    }

    status = xTaskCreate(consumer2_sem, "consumer", 200, (void*) semaphores, 3, NULL);
    if (status != pdPASS) {
        PRINTF("Task creation failed!.\r\n");
        while (1)
            ;
    }

    vTaskStartScheduler();

    while (1) {
    }
}
\end{lstlisting}
\end{filecontents}

\begin{filecontents}[overwrite]{./sections/p3_producer.tex}
\begin{lstlisting}[language=c,caption=Problem 3 Producer Task, label=list:p3_prod]
void producer_sem(void *pvParameters) {
    SemaphoreHandle_t *semaphores = (SemaphoreHandle_t*) pvParameters;
    SemaphoreHandle_t producer1_semaphore = semaphores[0];
    SemaphoreHandle_t producer2_semaphore = semaphores[1];
    SemaphoreHandle_t consumer_semaphore = semaphores[2];
    BaseType_t status1, status2;

    printf("please enter a string\n");
    scanf("%s", usr_str);

    while (1) {
        status1 = xSemaphoreTake(consumer_semaphore, portMAX_DELAY);
        status2 = xSemaphoreTake(consumer_semaphore, portMAX_DELAY);

        if (status1 != pdPASS || status2 != pdPASS) {
            PRINTF("Failed to acquire consumer_semaphore\r\n");
            while (1)
                ;
        }

        xSemaphoreGive(producer1_semaphore);
        xSemaphoreGive(producer2_semaphore);
        vTaskDelay(1000 / portTICK_PERIOD_MS);
    }

}    
\end{lstlisting}
\end{filecontents}


\begin{filecontents}[overwrite]{./sections/p3_consumer1.tex}
\begin{lstlisting}[language=c,caption=Problem 3 Consumer 1 Task, label=list:p3_cons1]
void consumer1_sem(void *pvParameters) {
    SemaphoreHandle_t *semaphores = (SemaphoreHandle_t*) pvParameters;
    SemaphoreHandle_t producer1_semaphore = semaphores[0];
    SemaphoreHandle_t consumer_semaphore = semaphores[2];
    BaseType_t status;

    while (1) {
        xSemaphoreGive(consumer_semaphore);
        status = xSemaphoreTake(producer1_semaphore, portMAX_DELAY);
        if (status != pdPASS) {
            PRINTF("Failed to acquire producer1_semaphore\r\n");
            while (1)
                ;
        }
        PRINTF("Received Value = %s\r\n", usr_str);
    }
}
\end{lstlisting}
\end{filecontents}

\begin{filecontents}[overwrite]{./sections/p3_consumer2.tex}
\begin{lstlisting}[language=c,caption=Problem 3 Consumer 2 Task, label=list:p3_cons2]
void consumer2_sem(void *pvParameters) {
    SemaphoreHandle_t *semaphores = (SemaphoreHandle_t*) pvParameters;
    SemaphoreHandle_t producer2_semaphore = semaphores[1];
    SemaphoreHandle_t consumer_semaphore = semaphores[2];
    BaseType_t status;

    char *receive_str_cap;

    receive_str_cap = malloc(sizeof(char) * 100);

    while (1) {
        xSemaphoreGive(consumer_semaphore);
        status = xSemaphoreTake(producer2_semaphore, portMAX_DELAY);

        if (status != pdPASS) {
            PRINTF("Failed to acquire producer2_semaphore\r\n");

            while (1)
                ;
        }

        for (int i = 0; i < 100; i++) {
            receive_str_cap[i] = toupper(usr_str[i]);
        }

        PRINTF("Received Value in captal = %s\r\n", receive_str_cap);
    }
}
\end{lstlisting}
\end{filecontents}

\section*{Problem 3}
\begin{enumerate}
\item
It is possible to use a single producer semaphore to control the two consumers with a 2,2 counting semaphores, but you should initialize the semaphore to 1 to make sure only one consumer accesses it at a time. When the semaphore is set to 1. Then the other consumer has to wait until the one who holes it by releasing with xSemaphoreGive.

\item
For this problem, we used a similar structure as the experiment. Two producer semaphore and one consumer counting semaphore which counts to 2 and is initialized to 2. The producer will take consumer semaphore and then give semaphore back to them. The user string is taken from the console as previous questions. When the producer task receives a string, it will expect a signal from both consumer tasks, then will signal both producer semaphores. The first consumer will print the string as it is, and the second consumer will change all the letters to the capital letters with the \texttt{toupper} function before printing the string.

The main function is shown below in Listing~\ref{list:p3_main}. The tasks are shown in Listings~\ref{list:p3_prod} to \ref{list:p3_cons2}.

\begin{lstlisting}[language=c,caption=Problem 3 main, label=list:p3_main]
int main(void) {
    char txbuff[] = "Hello World\r\n";

    int motorInput = 0;
    int servoInput = 0;
    float motorDutyCycle;
    float servoDutyCycle;
    /* Init board hardware. */
    BOARD_InitBootPins();
    BOARD_InitBootClocks();
    BOARD_InitDebugConsole();

    setupUART();
    setupPWM();

    /******* Delay *******/
    for (volatile int i = 0U; i < 1000000; i++)
        __asm("NOP");

    PRINTF("%s", txbuff);

    UART_WriteBlocking(TARGET_UART, txbuff, sizeof(txbuff) - 1);

    while (1) {
        if (new_char == 7) {
            new_char = 0;

            PRINTF("outside the interrupt buffer = %s \n ", buffer);

            PRINTF("cleared interrupt\n");

            sscanf(buffer, "%4d%3d", &motorInput, &servoInput);
            PRINTF("GOT input as motorInput = %d, servoInput = %d", motorInput, servoInput);
            motorDutyCycle = motorInput * 0.025f / 100.0f + 0.070745;
            servoDutyCycle = servoInput * 0.025f / 45.0f + 0.078;

            updatePWM_dutyCycle(FTM_CHANNEL_SERVO_MOTOR, servoDutyCycle);
            updatePWM_dutyCycle(FTM_CHANNEL_DC_MOTOR, motorDutyCycle);

            FTM_SetSoftwareTrigger(FTM_MOTOR, true);
            FTM_SetSoftwareTrigger(FTM_MOTOR, true);
        }
    }
}
\end{lstlisting}
%% LaTeX2e file `./sections/p3_producer.tex'
%% generated by the `filecontents' environment
%% from source `lab2' on 2023/03/03.
%%
\begin{lstlisting}[language=c,caption=Problem 3 Producer Task, label=list:p3_prod]
void producer_sem(void *pvParameters) {
    SemaphoreHandle_t *semaphores = (SemaphoreHandle_t*) pvParameters;
    SemaphoreHandle_t producer1_semaphore = semaphores[0];
    SemaphoreHandle_t producer2_semaphore = semaphores[1];
    SemaphoreHandle_t consumer_semaphore = semaphores[2];
    BaseType_t status1, status2;

    printf("please enter a string\n");
    scanf("%s", usr_str);

    while (1) {
        status1 = xSemaphoreTake(consumer_semaphore, portMAX_DELAY);
        status2 = xSemaphoreTake(consumer_semaphore, portMAX_DELAY);

        if (status1 != pdPASS || status2 != pdPASS) {
            PRINTF("Failed to acquire consumer_semaphore\r\n");
            while (1)
                ;
        }

        xSemaphoreGive(producer1_semaphore);
        xSemaphoreGive(producer2_semaphore);
        vTaskDelay(1000 / portTICK_PERIOD_MS);
    }

}
\end{lstlisting}

%% LaTeX2e file `./sections/p3_consumer1.tex'
%% generated by the `filecontents' environment
%% from source `lab2' on 2023/03/03.
%%
\begin{lstlisting}[language=c,caption=Problem 3 Consumer 1 Task, label=list:p3_cons1]
void consumer1_sem(void *pvParameters) {
    SemaphoreHandle_t *semaphores = (SemaphoreHandle_t*) pvParameters;
    SemaphoreHandle_t producer1_semaphore = semaphores[0];
    SemaphoreHandle_t consumer_semaphore = semaphores[2];
    BaseType_t status;

    while (1) {
        xSemaphoreGive(consumer_semaphore);
        status = xSemaphoreTake(producer1_semaphore, portMAX_DELAY);
        if (status != pdPASS) {
            PRINTF("Failed to acquire producer1_semaphore\r\n");
            while (1)
                ;
        }
        PRINTF("Received Value = %s\r\n", usr_str);
    }
}
\end{lstlisting}

%% LaTeX2e file `./sections/p3_consumer2.tex'
%% generated by the `filecontents' environment
%% from source `lab2' on 2023/03/03.
%%
\begin{lstlisting}[language=c,caption=Problem 3 Consumer 2 Task, label=list:p3_cons2]
void consumer2_sem(void *pvParameters) {
    SemaphoreHandle_t *semaphores = (SemaphoreHandle_t*) pvParameters;
    SemaphoreHandle_t producer2_semaphore = semaphores[1];
    SemaphoreHandle_t consumer_semaphore = semaphores[2];
    BaseType_t status;

    char *receive_str_cap;

    receive_str_cap = malloc(sizeof(char) * 100);

    while (1) {
        xSemaphoreGive(consumer_semaphore);
        status = xSemaphoreTake(producer2_semaphore, portMAX_DELAY);

        if (status != pdPASS) {
            PRINTF("Failed to acquire producer2_semaphore\r\n");

            while (1)
                ;
        }

        for (int i = 0; i < 100; i++) {
            receive_str_cap[i] = toupper(usr_str[i]);
        }

        PRINTF("Received Value in captal = %s\r\n", receive_str_cap);
    }
}
\end{lstlisting}


\end{enumerate}