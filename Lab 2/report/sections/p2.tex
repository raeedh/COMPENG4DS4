\begin{filecontents}[overwrite]{./sections/p2_main.tex}
\begin{lstlisting}[language=c,caption=Problem 2 main, label=list:p2_main]
int main(void) {
    BaseType_t status;
    /* Init board hardware. */
    BOARD_InitBootPins();
    BOARD_InitBootClocks();
    BOARD_InitDebugConsole();

    queue1 = xQueueCreate(10, sizeof(char*));

    if (queue1 == NULL) {
        PRINTF("Queue creation failed!.\r\n");
        while (1)
            ;
    }

    status = xTaskCreate(producer_queue, "producer", 200, (void*) queue1, 2, NULL);
    if (status != pdPASS) {
        PRINTF("Task creation failed!.\r\n");
        while (1)
            ;
    }

    status = xTaskCreate(consumer_queue, "consumer", 200, (void*) queue1, 3, NULL);
    if (status != pdPASS) {
        PRINTF("Task creation failed!.\r\n");
        while (1)
            ;
    }

    vTaskStartScheduler();

    while (1) {
    }

}
\end{lstlisting}
\end{filecontents}

\begin{filecontents}[overwrite]{./sections/p2_producer.tex}
\begin{lstlisting}[language=c,caption=Problem 2 Producer Queue, label=list:p2_prod]
void producer_queue(void *pvParameters) {
    QueueHandle_t queue1 = (QueueHandle_t) pvParameters;
    BaseType_t status;

    char *usr_str;
    usr_str = malloc(sizeof(char) * 10);

    printf("please enter a string\n");
    scanf("%s", usr_str);

    while (1) {
        status = xQueueSend(queue1, (void* ) &usr_str, portMAX_DELAY);
        if (status != pdPASS) {
            PRINTF("Queue Send failed!.\r\n");
            while (1)
                ;
        }

        vTaskDelay(1000 / portTICK_PERIOD_MS);
    }

    vTaskDelete(NULL);
}
\end{lstlisting}
\end{filecontents}


\begin{filecontents}[overwrite]{./sections/p2_consumer.tex}
\begin{lstlisting}[language=c,caption=Problem 2 Consumer Queue, label=list:p2_cons]
void consumer_queue(void *pvParameters) {
    QueueHandle_t queue1 = (QueueHandle_t) pvParameters;
    BaseType_t status;

    char *receive_str;

    while (1) {
        status = xQueueReceive(queue1, (void*) &receive_str, portMAX_DELAY);

        if (status != pdPASS) {
            PRINTF("Queue Receive failed!.\r\n");

            while (1)
                ;
        }

        PRINTF("Received Value = %s\r\n", receive_str);
    }
}    
\end{lstlisting}
\end{filecontents}

\section*{Problem 2}
When repeating Problem 1 with queues, we have a producer task handling the user input and a consumer task printing the string. The priority of the producer is 2 and the consumer is 3. The queue declared is \texttt{queue1}. In the producer task, the user input \texttt{usr\_str} will be put in the queue by \texttt{xQueueSend}. The consumer task is blocked by \texttt{xQueueReceive} until producer one sends the input string, which will also be received by \texttt{xQueueReceive}. After the string is received on the consumer side, it will be printed to the console.

The main function is shown below in Listing~\ref{list:p2_main}. The two tasks are shown in Listings~\ref{list:p2_prod} and \ref{list:p2_cons}.

%% LaTeX2e file `./sections/p2_main.tex'
%% generated by the `filecontents' environment
%% from source `lab2' on 2023/03/03.
%%
\begin{lstlisting}[language=c,caption=Problem 2 main, label=list:p2_main]
int main(void) {
    BaseType_t status;
    /* Init board hardware. */
    BOARD_InitBootPins();
    BOARD_InitBootClocks();
    BOARD_InitDebugConsole();

    queue1 = xQueueCreate(10, sizeof(char*));

    if (queue1 == NULL) {
        PRINTF("Queue creation failed!.\r\n");
        while (1)
            ;
    }

    status = xTaskCreate(producer_queue, "producer", 200, (void*) queue1, 2, NULL);
    if (status != pdPASS) {
        PRINTF("Task creation failed!.\r\n");
        while (1)
            ;
    }

    status = xTaskCreate(consumer_queue, "consumer", 200, (void*) queue1, 3, NULL);
    if (status != pdPASS) {
        PRINTF("Task creation failed!.\r\n");
        while (1)
            ;
    }

    vTaskStartScheduler();

    while (1) {
    }

}
\end{lstlisting}

%% LaTeX2e file `./sections/p2_producer.tex'
%% generated by the `filecontents' environment
%% from source `lab2' on 2023/03/03.
%%
\begin{lstlisting}[language=c,caption=Problem 2 Producer Queue, label=list:p2_prod]
void producer_queue(void *pvParameters) {
    QueueHandle_t queue1 = (QueueHandle_t) pvParameters;
    BaseType_t status;

    char *usr_str;
    usr_str = malloc(sizeof(char) * 10);

    printf("please enter a string\n");
    scanf("%s", usr_str);

    while (1) {
        status = xQueueSend(queue1, (void* ) &usr_str, portMAX_DELAY);
        if (status != pdPASS) {
            PRINTF("Queue Send failed!.\r\n");
            while (1)
                ;
        }

        vTaskDelay(1000 / portTICK_PERIOD_MS);
    }

    vTaskDelete(NULL);
}
\end{lstlisting}

%% LaTeX2e file `./sections/p2_consumer.tex'
%% generated by the `filecontents' environment
%% from source `lab2' on 2023/03/03.
%%
\begin{lstlisting}[language=c,caption=Problem 2 Consumer Queue, label=list:p2_cons]
void consumer_queue(void *pvParameters) {
    QueueHandle_t queue1 = (QueueHandle_t) pvParameters;
    BaseType_t status;

    char *receive_str;

    while (1) {
        status = xQueueReceive(queue1, (void*) &receive_str, portMAX_DELAY);

        if (status != pdPASS) {
            PRINTF("Queue Receive failed!.\r\n");

            while (1)
                ;
        }

        PRINTF("Received Value = %s\r\n", receive_str);
    }
}
\end{lstlisting}
