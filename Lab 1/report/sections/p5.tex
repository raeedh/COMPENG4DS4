\section*{Problem 5}
The functions from Problem 4 and Experiment 4: Part B were used to convert the e-compass SDK example to use SPI instead of the default I2C. The following steps were taken to convert the project to SPI:
\begin{enumerate}
    \item Add dspi driver to the project.
    \item Use the same code of the functions BOARD InitBootPins and BOARD InitBootClocks from the project in Experiment 4: Part A.
    \item Remove the following functions from ecompass\_peripheral.c
    \begin{itemize}
        \item \texttt{i2c\_release\_bus\_delay}
        \item \texttt{BOARD\_I2C\_ReleaseBus}
    \end{itemize}
    \item Add the following functions
    \begin{itemize}
        \item \texttt{setupSPI}
        \item \texttt{voltageRegulatorEnable}
        \item \texttt{accelerometerEnable}
        \item \texttt{SPI\_read}
        \item \texttt{SPI\_write}
    \end{itemize}
    \item Update \texttt{fsl\_fxos.h} with the "typedef"s for \texttt{SPI\_WriteFunc\_t} and \texttt{SPI\_ReadFunc\_t}, and update the \texttt{fxos\_handle\_t} and \texttt{fxos\_config\_t} structures to point to the new SPI functions.
    \item Update \texttt{fsl\_fxos.c} to replace any references to \texttt{I2C\_SendFunc} and \texttt{I2C\_ReceiveFunc} with \texttt{SPI\_writeFunc} and \texttt{SPI\_readFunc}. Make sure to update the function arguments for when they appear in \texttt{FXOS\_ReadReg} and \texttt{FXOS\_WriteReg}.
    \item Update the main function similarly to the updates made in Experiment 4: Part B to configure SPI functions instead of I2C. The code listing for the main function is shown below in Listing~\ref{list:p5_main}.
    \begin{lstlisting}[language=c,caption=Problem 5 main, label=list:p5_main]
int main(void)
{
    fxos_config_t config = {0};
    
    uint16_t i              = 0;
    uint16_t loopCounter    = 0;
    double sinAngle         = 0;
    double cosAngle         = 0;
    double Bx               = 0;
    double By               = 0;
    uint8_t array_addr_size = 0;

    status_t result;
    
    /* Board pin, clock, debug console init */
    BOARD_InitBootPins();
    BOARD_InitBootClocks();

    voltageRegulatorEnable();
    accelerometerEnable();

    setupSPI();

    HW_Timer_init();

    /***** Delay *****/
    for (volatile uint32_t i = 0; i < 4000000; i++)
        __asm("NOP");


    /* Configure the SPI function */
    config.SPI_writeFunc = SPI_write;
    config.SPI_readFunc = SPI_read;

    result = FXOS_Init(&g_fxosHandle, &config);
    if (kStatus_Success != result)
    {
        PRINTF("\r\nSensor device initialize failed!\r\n");
    }

    /* Get sensor range */
    if (kStatus_Success != FXOS_ReadReg(&g_fxosHandle, XYZ_DATA_CFG_REG, &g_sensorRange, 1))
    {
        PRINTF("\r\nGet sensor range failed!\r\n");
    }

    switch (g_sensorRange)
    {
        case 0x00:
            g_dataScale = 2U;
            break;
        case 0x01:
            g_dataScale = 4U;
            break;
        case 0x10:
            g_dataScale = 8U;
            break;
        default:
            break;
    }

    PRINTF("\r\nTo calibrate Magnetometer, roll the board on all orientations to get max and min values\r\n");
    PRINTF("\r\nPress any key to start calibrating...\r\n");
    GETCHAR();
    Magnetometer_Calibrate();

    /* Infinite loops */
    for (;;)
    {
        if (SampleEventFlag == 1) /* Fix loop */
        {
            SampleEventFlag = 0;
            g_Ax_Raw        = 0;
            g_Ay_Raw        = 0;
            g_Az_Raw        = 0;
            g_Ax            = 0;
            g_Ay            = 0;
            g_Az            = 0;
            g_Mx_Raw        = 0;
            g_My_Raw        = 0;
            g_Mz_Raw        = 0;
            g_Mx            = 0;
            g_My            = 0;
            g_Mz            = 0;

            /* Read sensor data */
            Sensor_ReadData(&g_Ax_Raw, &g_Ay_Raw, &g_Az_Raw, &g_Mx_Raw, &g_My_Raw, &g_Mz_Raw);

            /* Average accel value */
            for (i = 1; i < MAX_ACCEL_AVG_COUNT; i++)
            {
                g_Ax_buff[i] = g_Ax_buff[i - 1];
                g_Ay_buff[i] = g_Ay_buff[i - 1];
                g_Az_buff[i] = g_Az_buff[i - 1];
            }

            g_Ax_buff[0] = g_Ax_Raw;
            g_Ay_buff[0] = g_Ay_Raw;
            g_Az_buff[0] = g_Az_Raw;

            for (i = 0; i < MAX_ACCEL_AVG_COUNT; i++)
            {
                g_Ax += (double)g_Ax_buff[i];
                g_Ay += (double)g_Ay_buff[i];
                g_Az += (double)g_Az_buff[i];
            }

            g_Ax /= MAX_ACCEL_AVG_COUNT;
            g_Ay /= MAX_ACCEL_AVG_COUNT;
            g_Az /= MAX_ACCEL_AVG_COUNT;

            if (g_FirstRun)
            {
                g_Mx_LP = g_Mx_Raw;
                g_My_LP = g_My_Raw;
                g_Mz_LP = g_Mz_Raw;
            }

            g_Mx_LP += ((double)g_Mx_Raw - g_Mx_LP) * 0.01;
            g_My_LP += ((double)g_My_Raw - g_My_LP) * 0.01;
            g_Mz_LP += ((double)g_Mz_Raw - g_Mz_LP) * 0.01;

            /* Calculate magnetometer values */
            g_Mx = g_Mx_LP - g_Mx_Offset;
            g_My = g_My_LP - g_My_Offset;
            g_Mz = g_Mz_LP - g_Mz_Offset;

            /* Calculate roll angle g_Roll (-180deg, 180deg) and sin, cos */
            g_Roll   = atan2(g_Ay, g_Az) * RadToDeg;
            sinAngle = sin(g_Roll * DegToRad);
            cosAngle = cos(g_Roll * DegToRad);

            /* De-rotate by roll angle g_Roll */
            By   = g_My * cosAngle - g_Mz * sinAngle;
            g_Mz = g_Mz * cosAngle + g_My * sinAngle;
            g_Az = g_Ay * sinAngle + g_Az * cosAngle;

            /* Calculate pitch angle g_Pitch (-90deg, 90deg) and sin, cos*/
            g_Pitch  = atan2(-g_Ax, g_Az) * RadToDeg;
            sinAngle = sin(g_Pitch * DegToRad);
            cosAngle = cos(g_Pitch * DegToRad);

            /* De-rotate by pitch angle g_Pitch */
            Bx = g_Mx * cosAngle + g_Mz * sinAngle;

            /* Calculate yaw = ecompass angle psi (-180deg, 180deg) */
            g_Yaw = atan2(-By, Bx) * RadToDeg;
            if (g_FirstRun)
            {
                g_Yaw_LP   = g_Yaw;
                g_FirstRun = false;
            }

            g_Yaw_LP += (g_Yaw - g_Yaw_LP) * 0.01;

            if (++loopCounter > 10)
            {
                PRINTF("\r\nCompass Angle: %3.1lf", g_Yaw_LP);
                loopCounter = 0;
            }
        }
    } /* End infinite loops */
}
\end{lstlisting}
\end{enumerate}
