\begin{filecontents}[overwrite]{./sections/part2_cpp.tex}
\begin{lstlisting}[language=c++,caption=Part 2 C++ Code, label=list:part2_cpp]
#include <px4_platform_common/px4_config.h>
#include <px4_platform_common/log.h>

#include <drivers/drv_hrt.h>
#include <uORB/Publication.hpp>
#include <uORB/topics/test_motor.h>
#include <uORB/topics/debug_value.h>

#define DC_MOTOR 0
#define SERVO_MOTOR 1

extern "C" __EXPORT int hello_world_main(int argc, char *argv[]);

int hello_world_main(int argc, char *argv[])
{
    px4_sleep(2);

    debug_value_s debug_data;
    int debug_handle = orb_subscribe(ORB_ID(debug_value));
    orb_set_interval(debug_handle, 500);

    test_motor_s test_motor;
    double motor_value = 0; // a number between 0 to 1

    test_motor_s servo_motor;
    double angle_value = 0.5; // a number between 0 to 1

    uORB::Publication<test_motor_s> test_motor_pub(ORB_ID(test_motor));

    int dist = 0;
    int ret = 0;

    PX4_INFO("Motor speed is %f", motor_value);
    test_motor.timestamp = hrt_absolute_time();
    test_motor.motor_number = DC_MOTOR;
    test_motor.value = (float)motor_value;
    test_motor.action = test_motor_s::ACTION_RUN;
    test_motor.driver_instance = 0;
    test_motor.timeout_ms = 0;

    test_motor_pub.publish(test_motor);

    PX4_INFO("Servo angle is %f", angle_value);
    servo_motor.timestamp = hrt_absolute_time();
    servo_motor.motor_number = SERVO_MOTOR;
    servo_motor.value = (float)angle_value;
    servo_motor.action = test_motor_s::ACTION_RUN;
    servo_motor.driver_instance = 0;
    servo_motor.timeout_ms = 0;

    test_motor_pub.publish(servo_motor);

    while (1)
    {
        orb_copy(ORB_ID(debug_value), debug_handle, &debug_data);

        //get motor and angle values from rc_data
        if (debug_data.ind == 0) dist = debug_data.value;
        if (debug_data.ind == 1) ret = debug_data.value;

        // do something with dist and ret
        if (dist >= 50) {
            motor_value = 1;
        } else if ( dist >= 15 && dist < 50 ) {
            motor_value = 0.7;
        } else if (dist < 15) {
            motor_value = 0.5;
        }

        if (ret == 0) {
            angle_value = 1;
        } else if(ret == 1){
            angle_value = 0.5;
        } else if(ret == 2){
            angle_value = 0;
        }

        PX4_INFO("Motor speed is %f", motor_value);
        test_motor.timestamp = hrt_absolute_time();
        test_motor.motor_number = DC_MOTOR;
        test_motor.value = (float)motor_value;
        test_motor.action = test_motor_s::ACTION_RUN;
        test_motor.driver_instance = 0;
        test_motor.timeout_ms = 0;

        test_motor_pub.publish(test_motor);

        PX4_INFO("Servo angle is %f", angle_value);
        servo_motor.timestamp = hrt_absolute_time();
        servo_motor.motor_number = SERVO_MOTOR;
        servo_motor.value = (float)angle_value;
        servo_motor.action = test_motor_s::ACTION_RUN;
        servo_motor.driver_instance = 0;
        servo_motor.timeout_ms = 0;

        test_motor_pub.publish(servo_motor);

        px4_usleep(20000);
    }

    PX4_INFO("The motor will be stopped");
    test_motor.timestamp = hrt_absolute_time();
    test_motor.motor_number = DC_MOTOR;
    test_motor.value = 0.5;
    test_motor.driver_instance = 0;
    test_motor.timeout_ms = 0;

    test_motor_pub.publish(test_motor);

    PX4_INFO("The servo motor will be stopped");
    servo_motor.timestamp = hrt_absolute_time();
    servo_motor.motor_number = SERVO_MOTOR;
    servo_motor.value = 0.5;
    servo_motor.driver_instance = 0;
    servo_motor.timeout_ms = 0;

    test_motor_pub.publish(servo_motor);


    return 0;
}    
\end{lstlisting}
\end{filecontents}

\begin{filecontents}[overwrite]{./sections/part2_py.tex}
\begin{lstlisting}[language=python,caption=Part 2 Python Code, label=list:part2_py]
#Libraries
from pymavlink import mavutil
import RPi.GPIO as GPIO
import time
import cv2
import numpy as np
import math

###########################
# Ultrasonic Sensor Setup #
###########################

#set GPIO Pins
GPIO_TRIGGER = 23
GPIO_ECHO = 24


def ultrasonic_setup():
    #GPIO Mode (BOARD / BCM)
    GPIO.setmode(GPIO.BCM)

    #set GPIO direction (IN / OUT)
    GPIO.setup(GPIO_TRIGGER, GPIO.OUT)
    GPIO.setup(GPIO_ECHO, GPIO.IN)

def distance():
    # set Trigger to HIGH
    GPIO.output(GPIO_TRIGGER, True)

    # set Trigger after 0.01ms to LOW
    time.sleep(0.00001)
    GPIO.output(GPIO_TRIGGER, False)

    StartTime = time.time()
    StopTime = time.time()

    # save StartTime
    while GPIO.input(GPIO_ECHO) == 0:
        StartTime = time.time()

    # save time of arrival
    while GPIO.input(GPIO_ECHO) == 1:
        StopTime = time.time()

    # time difference between start and arrival
    TimeElapsed = StopTime - StartTime
    # multiply with the sonic speed (34300 cm/s)
    # and divide by 2, because there and back
    distance = (TimeElapsed * 34300) / 2

    return distance

################
# Camera Setup #
################

cap = cv2.VideoCapture(0)
StepSize = 5

def getChunks(l, n):
    """Yield successive n-sized chunks from l."""
    a = []
    for i in range(0, len(l), n):
        a.append(l[i:i + n])
    return a

def process_camera_frame():
    ret,frame = cap.read()
    img = cv2.flip(frame, 0)
    blur = cv2.bilateralFilter(img,9,40,40)
    edges = cv2.Canny(blur,50,100)
    img_h = img.shape[0] - 1
    img_w = img.shape[1] - 1
    EdgeArray = []

    for j in range(0,img_w,StepSize):
        pixel = (j,0)
        for i in range(img_h-5,0,-1):
            if edges.item(i,j) == 255:
                pixel = (j,i)
                EdgeArray.append(pixel)
                break

    if len(EdgeArray) != 0:
        chunks = getChunks(EdgeArray, math.ceil(len(EdgeArray)/3))
    else:
        return

    #c = []
    distance = []
    for i in range(len(chunks)):
        x_vals = []
        y_vals = []
        for (x,y) in chunks[i]:
            x_vals.append(x)
            y_vals.append(y)
        avg_x = int(np.average(x_vals))
        avg_y = int(np.average(y_vals))
        #c.append([avg_y,avg_x])
        distance.append(math.sqrt((avg_x - 320)**2 + (avg_y - 640)**2))
        cv2.line(img, (320, 640), (avg_x,avg_y), (0,0,255), 2)

    cv2.imshow("frame", img)
    cv2.waitKey(5)
    if(distance[0] < distance[1]):
        if(distance[0] < distance[2]):
            return 0
        else:
            return 2
    else:
        if(distance[1] < distance[2]):
            return 1
        else:
            return 2

##################################################

ultrasonic_setup();

# Start a connection
the_connection = mavutil.mavlink_connection('/dev/ttyACM0')

# Wait for the first heartbeat
# This sets the system and component ID of remote system for the link
the_connection.wait_heartbeat()
print("Heartbeat from system (system %u component %u)" % (the_connection.target_system, the_connection.target_component))


# Once connected, use 'the_connection' to get and send messages
value = 0

while True:
    dist = distance()
    print ("Measured Distance = %.1f cm" % dist)

    message = mavutil.mavlink.MAVLink_debug_message(0, 0, dist)
    the_connection.mav.send(message)

    ret = process_camera_frame()
    if ret == 0:
        print('Left direction is preferred')
    elif ret == 1:
        print('Forward direction is preferred')
    elif ret == 2:
        print('Right direction is preferred')

    message = mavutil.mavlink.MAVLink_debug_message(0, 1, ret)
    the_connection.mav.send(message)

    time.sleep(0.1)
    # time.sleep(1)
    print("Message sent")
\end{lstlisting}
\end{filecontents}

\section*{Part 2}

The sensor data is read from a python program on the Raspberry Pi. The python code for Part 2 is shown in Listing~\ref{list:part2_py}. The python program reads data from the ultrasonic sensor and camera using the algorithms provided in Experiments 5 and 6, and establishes a mavlink connection to send debug messages to the FMU (as shown in Experiment 4). The distance value is sent in a debug message with an index of 1, and the preferred direction is sent in a debug message with an index of 0. The debug messages are sent separately through mavlink as the uORB structure in the C++ code only contains one debug message, requiring two reads.

\input{sections/part2_py.tex}

The motor actuation is handled in the C++ code using the code from Part 1. The C++ code for Part 2 is shown in Listing~\ref{list:part2_cpp}.  The code is modified to read debug messages (as shown in Experiment 3C) instead of the RC values. When the debug messages are read by the C++ application, the index is used to determined if the data copied is the distance data or the preferred direction. The servo angle is set based on the preferred direction. The motor value is set based on the distance. The motor will stop if the distance is below 15cm, with operate at roughly 40\% speed when between 15cm and 50cm, and at 100\% speed above 50cm.

%% LaTeX2e file `./sections/part2_cpp.tex'
%% generated by the `filecontents' environment
%% from source `project2' on 2023/04/07.
%%
\begin{lstlisting}[language=c++,caption=Part 2 C++ Code, label=list:part2_cpp]
#include <px4_platform_common/px4_config.h>
#include <px4_platform_common/log.h>

#include <drivers/drv_hrt.h>
#include <uORB/Publication.hpp>
#include <uORB/topics/test_motor.h>
#include <uORB/topics/debug_value.h>

#define DC_MOTOR 0
#define SERVO_MOTOR 1

extern "C" __EXPORT int hello_world_main(int argc, char *argv[]);

int hello_world_main(int argc, char *argv[])
{
    px4_sleep(2);

    debug_value_s debug_data;
    int debug_handle = orb_subscribe(ORB_ID(debug_value));
    orb_set_interval(debug_handle, 500);

    test_motor_s test_motor;
    double motor_value = 0; // a number between 0 to 1

    test_motor_s servo_motor;
    double angle_value = 0.5; // a number between 0 to 1

    uORB::Publication<test_motor_s> test_motor_pub(ORB_ID(test_motor));

    int dist = 0;
    int ret = 0;

    PX4_INFO("Motor speed is %f", motor_value);
    test_motor.timestamp = hrt_absolute_time();
    test_motor.motor_number = DC_MOTOR;
    test_motor.value = (float)motor_value;
    test_motor.action = test_motor_s::ACTION_RUN;
    test_motor.driver_instance = 0;
    test_motor.timeout_ms = 0;

    test_motor_pub.publish(test_motor);

    PX4_INFO("Servo angle is %f", angle_value);
    servo_motor.timestamp = hrt_absolute_time();
    servo_motor.motor_number = SERVO_MOTOR;
    servo_motor.value = (float)angle_value;
    servo_motor.action = test_motor_s::ACTION_RUN;
    servo_motor.driver_instance = 0;
    servo_motor.timeout_ms = 0;

    test_motor_pub.publish(servo_motor);

    while (1)
    {
        orb_copy(ORB_ID(debug_value), debug_handle, &debug_data);

        //get motor and angle values from rc_data
        if (debug_data.ind == 0) dist = debug_data.value;
        if (debug_data.ind == 1) ret = debug_data.value;

        // do something with dist and ret
        if (dist >= 50) {
            motor_value = 1;
        } else if ( dist >= 15 && dist < 50 ) {
            motor_value = 0.7;
        } else if (dist < 15) {
            motor_value = 0.5;
        }

        if (ret == 0) {
            angle_value = 1;
        } else if(ret == 1){
            angle_value = 0.5;
        } else if(ret == 2){
            angle_value = 0;
        }

        PX4_INFO("Motor speed is %f", motor_value);
        test_motor.timestamp = hrt_absolute_time();
        test_motor.motor_number = DC_MOTOR;
        test_motor.value = (float)motor_value;
        test_motor.action = test_motor_s::ACTION_RUN;
        test_motor.driver_instance = 0;
        test_motor.timeout_ms = 0;

        test_motor_pub.publish(test_motor);

        PX4_INFO("Servo angle is %f", angle_value);
        servo_motor.timestamp = hrt_absolute_time();
        servo_motor.motor_number = SERVO_MOTOR;
        servo_motor.value = (float)angle_value;
        servo_motor.action = test_motor_s::ACTION_RUN;
        servo_motor.driver_instance = 0;
        servo_motor.timeout_ms = 0;

        test_motor_pub.publish(servo_motor);

        px4_usleep(20000);
    }

    PX4_INFO("The motor will be stopped");
    test_motor.timestamp = hrt_absolute_time();
    test_motor.motor_number = DC_MOTOR;
    test_motor.value = 0.5;
    test_motor.driver_instance = 0;
    test_motor.timeout_ms = 0;

    test_motor_pub.publish(test_motor);

    PX4_INFO("The servo motor will be stopped");
    servo_motor.timestamp = hrt_absolute_time();
    servo_motor.motor_number = SERVO_MOTOR;
    servo_motor.value = 0.5;
    servo_motor.driver_instance = 0;
    servo_motor.timeout_ms = 0;

    test_motor_pub.publish(servo_motor);


    return 0;
}
\end{lstlisting}
