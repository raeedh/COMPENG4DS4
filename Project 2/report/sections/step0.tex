\begin{filecontents}[overwrite]{./sections/step0_list.tex}
\begin{lstlisting}[language=c++,caption=Step 0 Code, label=list:step0]
#include <px4_platform_common/px4_config.h>
#include <px4_platform_common/log.h>

#include <uORB/topics/rc_channels.h>

extern "C" __EXPORT int hello_world_main(int argc, char *argv[]);

int hello_world_main(int argc, char *argv[])
{
    int rc_channel_handle;
    rc_channels_s rc_channel_data;

    rc_channel_handle = orb_subscribe(ORB_ID(rc_channels));
    orb_set_interval(rc_channel_handle, 200);

    while (1)
    {
        orb_copy(ORB_ID(rc_channels), rc_channel_handle, &rc_channel_data);

        PX4_INFO("header = %f, ch1 = %f, ch2 = %f, ch3 = %f, ch4 = %f, ch5 = %f, ch6 = %f, ch7 = %f, ch8 = %f",
                double(rc_channel_data.channels[0]),
                double(rc_channel_data.channels[1]),
                double(rc_channel_data.channels[2]),
                double(rc_channel_data.channels[3]),
                double(rc_channel_data.channels[4]),
                double(rc_channel_data.channels[5]),
                double(rc_channel_data.channels[6]),
                double(rc_channel_data.channels[7]),
                double(rc_channel_data.channels[8]));

        px4_usleep(200000);
    }

    return 0;
}
\end{lstlisting}
\end{filecontents}

\section*{Step 0}

To complete this task, we modify the code provided in Experiment 3B. To read the values of the RC channels, we subscribe to \texttt{rc\_channels} messages instead of \texttt{sensored\_combined} messages. The \texttt{rc\_channel\_s} structure provided in \texttt{rc\_channels} uORB topic directory contains the member \texttt{channels[18]}, which will contain the channel data. In this application, we access the data in these channels. The data contained in \texttt{rc\_channel\_data.channels} ranges from -1 to 1. The code for Step 0 is shown below in Listing~\ref{list:step0}.


%% LaTeX2e file `./sections/step0_list.tex'
%% generated by the `filecontents' environment
%% from source `project2' on 2023/04/07.
%%
\begin{lstlisting}[language=c++,caption=Step 0 Code, label=list:step0]
#include <px4_platform_common/px4_config.h>
#include <px4_platform_common/log.h>

#include <uORB/topics/rc_channels.h>

extern "C" __EXPORT int hello_world_main(int argc, char *argv[]);

int hello_world_main(int argc, char *argv[])
{
    int rc_channel_handle;
    rc_channels_s rc_channel_data;

    rc_channel_handle = orb_subscribe(ORB_ID(rc_channels));
    orb_set_interval(rc_channel_handle, 200);

    while (1)
    {
        orb_copy(ORB_ID(rc_channels), rc_channel_handle, &rc_channel_data);

        PX4_INFO("header = %f, ch1 = %f, ch2 = %f, ch3 = %f, ch4 = %f, ch5 = %f, ch6 = %f, ch7 = %f, ch8 = %f",
                double(rc_channel_data.channels[0]),
                double(rc_channel_data.channels[1]),
                double(rc_channel_data.channels[2]),
                double(rc_channel_data.channels[3]),
                double(rc_channel_data.channels[4]),
                double(rc_channel_data.channels[5]),
                double(rc_channel_data.channels[6]),
                double(rc_channel_data.channels[7]),
                double(rc_channel_data.channels[8]));

        px4_usleep(200000);
    }

    return 0;
}
\end{lstlisting}
